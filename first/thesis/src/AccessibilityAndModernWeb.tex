% !TEX encoding = IsoLatin2  % notwendige Zeile f"ur Mac-Benutzer (muss als Kommentar stehen); Windows-Benutzer k"onnen die Zeile l"oschen.

% LaTeX-Vorlage Version 3.1,  Juli 2011
% erstellt von Dr. Andreas Drauschke (andreas.drauschke@technikum-wien.at) und Dr. Susanne Teschl (susanne.teschl@technikum-wien.at)
% geringf"ugig adaptiert von Harald Stockinger (harald.stockinger@technikum-wien.at)

 
\documentclass[a4paper,bibtotoc,oneside]{scrbook} 
% F"ur kurze Arbeiten w"are auch die Dokumentklasse "scrartcl" ausreichend. In diesem Fall ist "section" die h"ochste Ebene ("chapter" gibt es dann nicht).
% \documentclass[a4paper,bibtotoc,oneside]{scrartcl}


% verlinkte Querverweise im pdf
\usepackage{hyperref}

% deutsche Anpassungen
\usepackage[ansinew]{inputenc}
\usepackage[T1]{fontenc}
\usepackage[ngerman]{babel}

% mathematische Symbole
\usepackage{amsmath,amssymb,amsfonts,amstext}

% Kopfzeilen frei gestaltbar
\usepackage{fancyhdr}
\lfoot[\fancyplain{}{}]{\fancyplain{}{}}
\rfoot[\fancyplain{}{}]{\fancyplain{}{}}
\cfoot[\fancyplain{}{\footnotesize\thepage}]{\fancyplain{}{\footnotesize\thepage}}
\lhead[\fancyplain{}{\footnotesize\nouppercase\leftmark}]{\fancyplain{}{}}
\chead{}
\rhead[\fancyplain{}{}]{\fancyplain{}{\footnotesize\nouppercase\sc\leftmark}} 

% Farben im Dokument m"oglich
\usepackage{color}

% Schriftart Helvetica
\usepackage{helvet}
\renewcommand{\familydefault}{cmss} 

% Graphiken einbinden: hier f"ur pdflatex
\usepackage[pdftex]{graphicx}

\usepackage{array}

% H"ohe und Breite des Textk"orpers etwas gr"osser definieren
\setlength{\textheight}{225mm}
\setlength{\textwidth}{1.05\textwidth}

% weniger Warnungen wegen "uberf"ullter Boxen
\tolerance = 9999
\sloppy

% Anpassung einiger "Uberschriften 
\renewcommand\figurename{Abbildung}
\renewcommand\tablename{Tabelle}

\begin{document}

% Kopf- und Fusszeilen initiieren
\pagestyle{fancy}
\pagenumbering{Alph}

% Deckblatt:
\thispagestyle{empty}
\begin{picture}(0,0)
\color{white}\sffamily
\put(-101,-749){\includegraphics[width=1.002\paperwidth, height=\paperheight]{BM_2011.pdf}}
\put(220,-670){\includegraphics[width=0.5\textwidth]{FHTW_Logo_4c.pdf}}
\put(-30, -20){\bfseries\huge BACHELORARBEIT}
% Titel des Studienganges einf"ugen:
\put(-30,-50){\Large im Studiengang BIF}
% Titel der Arbeit einf"ugen:
% Die Minipage wird gesetzt, damit auch mehrzeilige Titel m"oglich werden.
\put(-32,-150){
\begin{minipage}{14cm}
\bfseries\huge Accessibility im Modern Web
\end{minipage}
}
% Name der Autorin/des Autors eingeben:
\put(-30,-250){\large Ausgef"uhrt von: Bernhard Posselt}
% Personenkennzeichen der Autorin/des Autors eingeben:
\put(-30,-270){\large Personenkennzeichen: 1010257029}
% Name der Begutachterin/des Begutachters eingeben:
\put(-30,-310){\large Begutachter: Dipl.-Ing. Mag. Dr. Michael Tesar}
\put(-30,-350){\large Wien, \today} % das Datum des letzten Kompilierens wird automatisch eingesetzt
\color{black}
\end{picture}

\newpage


\section*{Eidesstattliche Erkl"arung}\thispagestyle{empty}
\glqq Ich erkl"are hiermit an Eides statt, dass ich die vorliegende Arbeit selbst"andig angefertigt habe. 
Die aus fremden Quellen direkt oder indirekt "ubernommenen Gedanken sind als solche kenntlich gemacht. 
Die Arbeit wurde bisher weder in gleicher noch in "ahnlicher Form einer anderen Pr"ufungsbeh"orde vorgelegt
und auch noch nicht ver"offentlicht. Ich versichere, dass die abgegebene Version jener im Uploadtool entspricht.\grqq\\[5\baselineskip]
\rule{5cm}{0.2pt}\hfill\rule{5cm}{0.2pt}\\
\phantom{Datum }Ort, Datum\hfill Unterschrift\hspace{15mm}

\newpage


\section*{Kurzfassung}\thispagestyle{empty}
In dieser Arbeit wird ein "Uberblick "uber moderne Web-Technologien im Bereich
Accessibility gegeben. Immer mehr "offentliche Einrichtungen stellen heutzutage
Zugang zu wichtige Informationen und Services im Web bereit. Dadurch muss dieser
Zugang universell erreichbar und nutzbar sein. Um eine L"osung f"ur dieses
Problem zu finden, wird ein Blick auf die Empfehlungen der W3C geworfen. Auch
auf neuere Techniken wie ARIA, die erst in einer Candidate Recommendation
vorhanden sind, werden eingangen. Das Ziel dieser Arbeit sollte ein guten
"Uberblick "uber den heutigen Stand bieten und die Vorteile der Nutzung von
Accessibility Methoden hervorstreichen. 
\vfill
\paragraph*{Schlagw"orter:} Accessibility, Web, ARIA, HTML5


\newpage

\section*{Abstract}\thispagestyle{empty}
This thesis will present an overview over accessibility techniques used in the
modern web. Over the last years the usage of the web to present and
access information and services of public institutions has increased
dramatically. This requires the services and information to be accessible for
every citizen. To find a solution for this problem, this thesis will look at
the W3C's recommendations and guidelines. Also newer techniques like ARIA
and the role attribute which are currently in the Candidate Recommendation
phase will be part of it. The goal of this thesis is to present the current
status of accessibility in the web and to show the advantages of using these
techniques.
\vfill
\paragraph*{Keywords:} Accessibility, Web, ARIA, HTML5
\newpage

%\section*{Danksagung}
%\thispagestyle{empty}
%Text Text Text Text Text Text Text Text Text Text Text Text Text Text Text Text
%\newpage

\tableofcontents\thispagestyle{empty}
\newpage

\pagenumbering{arabic}
\setcounter{page}{1}

% Falls die Kapitel"uberschriften zu lang f"ur die Kopfzeile oder das Inhaltsverzeichnis sind, so erzielt man
% dort Kurzformen der Kapitelbezeichnungen mittels:
% \chapter[Kurzform]{Lange "Uberschrift}
\chapter{Einf"uhrung}
Das Web spielt im "offentlichen Bereich eine immer gr"o"sere Rolle. Viele
Services und Informationen werden bereits "uber eigene Portale angeboten.
Dadurch wird das "offentliche Leben immer mehr mit dem Web verwoben und es muss
ein universeller Zugang f"ur alle B"urger m"oglich sein. Darunter fallen auch
Personen, welche mit k"orperlichen oder geistigen Einschr"ankungen leben
m"ussen.

Diese Gruppe macht nicht nur einen kleinen Prozentteil in der Bev"olkerung aus.
Laut einem Bericht des U.S. Census Bureau cite aus dem Jahre 2000
leben alleine in den USA ca. 49.7 Millionen Menschen mit k"orperlichen oder
geistigen Einschr"ankungen (entspricht ca. 20\% der amerikanischen
Bev"olkerung), davon haben 42.9 Millionen eine schwere Einschr"ankung und 6.8
Millionen haben eine so gravierende Einschr"ankung, dass sie Hilfe in ihrem
allt"aglichem Leben brauchen \cite{us_cens}. Laut der World Health Organization
wird die weltweite Anzahl von Personen mit Einschr"ankungen auf
500 bis 600 Millionen Menschen gesch"atzt \cite{who_dis}.

Auch die UN erkennt eine immer gr"o"ser werdene Wichtigkeit in der angemessenen
Berteitstellungen von Informationen f"ur k"orperlich oder geistig
engeschr"ankten Personen. Dies spiegelt sich in der UN Convention of Rights
for Persons with Disabilities wieder, welche am 30. M"arz 2007 unterzeichnet
wurde. Sie erhielt die \glqq meisten Unterschriften an einem Er"offnungstag in
der Geschichte der UN Konventionen\grqq \cite{un_disabilities} 

Jedoch kann es nicht nur aus den oben genannten Gr"unden erforderlich sein,
einen barrierefreien Zugang bereitzustellen, sondern sogar gesetzlich
vorgeschrieben. In den USA beispielsweise, gibt es daf"ur ein eigenes
Gesetzt, den the Americans with Disabilities Act aus dem Jahre 1990 (ADA) und
die Sektionen 504 und 508 aus dem Rehabilitation
Act (1973) \cite{achieving_web_acc}.

\section{Arten von k"orperlichen oder geistigen
Einschr"ankungen und assistive Technologien}

\subsection{Personen mit Sehst"orungen}
Zusammen mit den H"ohrst"orungen macht diese Gruppe laut dem U.S. Census Report
aus dem Jahre 2000 9,3 Millionen Personen aus. Das sind ca. 3,6 Prozent der
damaligen Bev"olkerung der USA \cite{us_cens}. 
Die Anzahl der Personen, die Probleme beim Lesen von Displays
haben, wird weltweit auf 6 Millionen Personen gesch"atzt
\cite{screen_read_frust}, die Anzahl blinder Personen in den USA wird auf 1
Million, weltweit auf 38 Millionen Personen gesch"atzt.
\cite{screen_read}.

Die meistgenutzten assistiven Technologien f"ur diese Personen sind
Screen-Reader und Braillen. Jedoch ziehen die Meisten den Screen-Reader
vor, weil nur wenige die Braille Schrift gut genug beherrschen. Au"serdem sind
die meisten spezialisierten Ger"ate teuer, w"ahrend Screen-Reader keine
zus"atzliche Hardware ben"otigen, weil sie als Software vorhanden sind
\cite{screen_read_frust}. Bestimmte Screen Reader wie GNOME Orca \cite{orca}
sind sogar frei verf"ugbar. Die gr"o"sten Probleme f"ur sehbehinderte Personen
entstehen nicht aus den Eingabeger"aten - die meisten Nutzer beherrschen
das 10-Finger System - sondern aus den Ausgabeger"aten \cite{screen_read_frust}.


\subsection{Personen mit H"ohst"orungen}

\subsection{Personen mit motorischen St"orungen}
Diese Gruppe macht laut dem U.S. Census Report 21.2 Millionen Personen aus. Das
sind ca. 8.2 Prozent der damaligen Bev"olkerung der USA \cite{us_cens}.

\subsection{Personen mit geistigen St"orungen}


\chapter{WCAG}
Success Criteria

\section{WCAG Version 1.0}

\section{WCAG Version 2.0}

\chapter{ARIA}

\chapter{Das Role Attribut}

\chapter{Schluss}
%\section[Erster Abschnitt]{"Uberschrift des ersten Abschnitts}



%\subsection[Erster Unterabschnitt]{"Uberschrift des ersten Unterabschnitts}



%\subsubsection[Erster Unter-Unterabschnitt]{Und noch eine Ebene tiefer} 


%\\[2\baselineskip]
%Hier wird auf Abbildung~\ref{Abb1} verwiesen. 
%\begin{figure}[htbp]
%\centering
%\includegraphics[width=75mm]{Buchruecken}
%\caption[Beschriftung eines Buchr"uckens.]{Beispiel f"ur die Beschriftung eines
%Buchr"uckens.}\label{Abb1}
%\end{figure}
%%Tabelle~\ref{Tab1} ist ein Beispiel daf"ur, wie eine Tabelle aussehen k"onnte.
%\begin{table}[htbp]
%\centering
%\begin{tabular}{ | c | c | c | }\hline
%{\bf Datum} & {\bf Thema} & {\bf Raum}\\ \hline
%\hline
%20. 08. 2008 & Graphentheorie & HS 3.13\\ \hline
%01. 10. 2008 & Biomathematik & HS 1.05\\ \hline
%\end{tabular}
%\caption[Semesterplan "`Angewandte Mathematik"'.]{Beispiel f"ur einen
%Semesterplan "`Angewandte Mathematik"'.}\label{Tab1}
%\end{table}

%\noindent
%Nun ein Beispiel f"ur eine abgesetzte Formel:
%\begin{equation}
%x =  - \frac{p}{2} \pm \sqrt{\left(\frac{p}{2}\right)^2 - q}.
%\end{equation}
%Und eine mehrzeilige Formel:
%\begin{eqnarray}
%f(t)&=& t^2 \label{For1},\\
%g(t) &=& t-1.
%\end{eqnarray}
%Hier wird auf die Formel (\ref{For1}) verwiesen. \\

%\noindent
%So kann zum Beispiel ein \glqq Source-Code\grqq\  angegeben werden: 
%\begin{verbatim}
%for (i=1; i < 10; i++) {...} 
%\end{verbatim}

%\noindent
%Hier ist ein Hyperlink auf die  \href{http://www.technikum-wien.at}{Homepage}
%der FH Technikum Wien. Email-Adressen k"onnen so verlinkt werden:
%\href{mailto:homer.simpson@springfield.com}{\texttt{
%homer.simpson@springfield.com}}\\

%\noindent
%In der Bibliothek der Fachhochschule Technikum Wien gibt es verschiedene
%einf"uhrende B"ucher zum Thema \glqq \LaTeX \grqq, zum Beispiel \cite{kop05},
%\cite{wil06} oder \cite{mgb+05d} (deutsche Version) bzw. \cite{mgb+04e}
%(englische Version). Empfehlenswerte Skripten f"ur \LaTeX-Einsteiger sind z.B.
%\cite{mj00} und \cite{mj95}. Sie sind frei im Internet verf"ugbar.



% Literaturverzeichnis
% Das Literaturverzeichnis kann auch nach einem allf"alligen Anhang positiioniert werden (siehe "`Leitfaden f"ur Bachelor- und Diplomarbeiten"', Version 2.0, Abschnitt 2.9).

% M"oglichkeit 1: Erzeugung des Literaturverzeichnisses mit BibTeX:
% Die Quellen sind in der Datei *.bib (hier Literatur.bib) einzugeben. Danach muss diese Vorlage einmal geTeXt werden, dann BibTeX angewendet werden und 
% anschliessend nochmals zweimal geTeXt werden.
% Im Text erfolgt die Zitierung mit dem Anker-Schl"usselwort, z.B. \cite{kop05}.
\bibliographystyle{IEEEtran}
\bibliography{Literatur}

% M"oglichkeit 2: Erzeugung eines Literaturverzeichnisses ohne BibTeX:
%\begin{thebibliography}{99}
%\bibitem[kop05]{kop05}
%H.~Kopka, {\em LaTeX, Band 1: Einf"uhrung}, Pearson Studium, M"unchen, 3.~Auflage, 2005.
%\bibitem[knu98]{knu98}
%F.~Mittelbach, M.~Goossens, J.~Braams, D.~Carlisle, and Ch. Rowley, {\em The LaTeX Companion}, 
%Addison-Wesley, 2nd edition, 2004.
%\end{thebibliography}

% Abbildungsverzeichnis
%\listoffigures
%\addcontentsline{toc}{chapter}{Abbildungsverzeichnis} % f"ugt den Eintrag
%"Abbildungsverzeichnis" im Inhaltsverzeichnis hinzu
%\newpage

% Tabellenverzeichnis
%\listoftables 
%\addcontentsline{toc}{chapter}{Tabellenverzeichnis} % f"ugt den Eintrag
%"Tabellenverzeichnis" im Inhaltsverzeichnis hinzu
%\newpage

% Abk"urzungsverzeichnis
% Bei Verwendung der Dokumentklasse "scrartcl" ist der Befehlt \addchap{Abk"urzungsverzeichnis} durch 
% \addsec{Abk"urzungsverzeichnis} zu ersetzen
\addchap{Abk"urzungsverzeichnis}
\hspace{-17mm}\begin{tabular}{>{\raggedleft}p{0.2\linewidth} p{0.75\linewidth} p{0.1\linewidth}}
www & World Wide Web \\
W3C & World Wide Web Consortium\\
URL & Uniform Resource Locator\\
ARIA & Accessible Rich Internet Applications
\end{tabular}

% Anh"ange
%\begin{appendix}
%\chapter[Erster Anhang]{"Uberschrift des ersten Anhangs}

%Text Text Text Text Text Text Text Text Text Text Text Text Text Text Text Text
%\end{appendix}

\end{document}
