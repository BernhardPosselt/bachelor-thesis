% !TEX encoding = IsoLatin2  % notwendige Zeile f"ur Mac-Benutzer (muss als Kommentar stehen); Windows-Benutzer k"onnen die Zeile l"oschen.

% LaTeX-Vorlage Version 3.1,  Juli 2011
% erstellt von Dr. Andreas Drauschke (andreas.drauschke@technikum-wien.at) und Dr. Susanne Teschl (susanne.teschl@technikum-wien.at)
% geringf"ugig adaptiert von Harald Stockinger (harald.stockinger@technikum-wien.at)

 
\documentclass[a4paper,bibtotoc,oneside]{scrbook} 
% F"ur kurze Arbeiten w"are auch die Dokumentklasse "scrartcl" ausreichend. In diesem Fall ist "section" die h"ochste Ebene ("chapter" gibt es dann nicht).
% \documentclass[a4paper,bibtotoc,oneside]{scrartcl}

%\usepackage{cclicenses}

% verlinkte Querverweise im pdf
\usepackage{hyperref}

% deutsche Anpassungen
\usepackage[ansinew]{inputenc}
\usepackage[T1]{fontenc}
\usepackage[ngerman]{babel}

% mathematische Symbole
\usepackage{amsmath,amssymb,amsfonts,amstext}

% Kopfzeilen frei gestaltbar
\usepackage{fancyhdr}
\lfoot[\fancyplain{}{}]{\fancyplain{}{}}
\rfoot[\fancyplain{}{}]{\fancyplain{}{}}
\cfoot[\fancyplain{}{\footnotesize\thepage}]{\fancyplain{}{\footnotesize\thepage}}
\lhead[\fancyplain{}{\footnotesize\nouppercase\leftmark}]{\fancyplain{}{}}
\chead{}
\rhead[\fancyplain{}{}]{\fancyplain{}{\footnotesize\nouppercase\sc\leftmark}} 

% Farben im Dokument m"oglich
\usepackage{color}

% Schriftart Helvetica
\usepackage{helvet}
\renewcommand{\familydefault}{cmss} 

% Graphiken einbinden: hier f"ur pdflatex
\usepackage[pdftex]{graphicx}

\usepackage{array}

% H"ohe und Breite des Textk"orpers etwas gr"osser definieren
\setlength{\textheight}{225mm}
\setlength{\textwidth}{1.05\textwidth}

% weniger Warnungen wegen "uberf"ullter Boxen
\tolerance = 9999
\sloppy

% Anpassung einiger "Uberschriften 
\renewcommand\figurename{Abbildung}
\renewcommand\tablename{Tabelle}

\begin{document}

% Kopf- und Fusszeilen initiieren
\pagestyle{fancy}
\pagenumbering{Alph}

% Deckblatt:
\thispagestyle{empty}
\begin{picture}(0,0)
\color{white}\sffamily
\put(-101,-749){\includegraphics[width=1.002\paperwidth, height=\paperheight]{BM_2011.pdf}}
\put(220,-670){\includegraphics[width=0.5\textwidth]{FHTW_Logo_4c.pdf}}
\put(-30, -20){\bfseries\huge BACHELORARBEIT}
% Titel des Studienganges einf"ugen:
\put(-30,-50){\Large im Studiengang Bachelor Informatik}
% Titel der Arbeit einf"ugen:
% Die Minipage wird gesetzt, damit auch mehrzeilige Titel m"oglich werden.
\put(-32,-150){
\begin{minipage}{14cm}
\bfseries\huge Software-Test von Web-Applikationen
\end{minipage}
}
% Name der Autorin/des Autors eingeben:
\put(-30,-250){\large Ausgef"uhrt von: Bernhard Posselt}
% Personenkennzeichen der Autorin/des Autors eingeben:
\put(-30,-270){\large Personenkennzeichen: 1010257029}
% Name der Begutachterin/des Begutachters eingeben:
\put(-30,-310){\large Begutachter: MSc Benedikt Salzbrunn}
\put(-30,-350){\large Wien, \today} % das Datum des letzten Kompilierens wird automatisch eingesetzt
\color{black}
\end{picture}

\newpage


\section*{Eidesstattliche Erkl"arung}\thispagestyle{empty}
\glqq Ich erkl"are hiermit an Eides statt, dass ich die vorliegende Arbeit selbst"andig angefertigt habe. 
Die aus fremden Quellen direkt oder indirekt "ubernommenen Gedanken sind als solche kenntlich gemacht. 
Die Arbeit wurde bisher weder in gleicher noch in "ahnlicher Form einer anderen Pr"ufungsbeh"orde vorgelegt
und auch noch nicht ver"offentlicht. Ich versichere, dass die abgegebene Version jener im Uploadtool entspricht.\grqq\\[5\baselineskip]
\rule{5cm}{0.2pt}\hfill\rule{5cm}{0.2pt}\\
\phantom{Datum }Ort, Datum\hfill Unterschrift\hspace{15mm}

\newpage


\section*{Kurzfassung}\thispagestyle{empty}

\vfill
\paragraph*{Schlagw"orter:} 


\newpage

\section*{Abstract}\thispagestyle{empty}

\vfill
\paragraph*{Keywords:}
\newpage

%\section*{Danksagung}
%\thispagestyle{empty}
%Text Text Text Text Text Text Text Text Text Text Text Text Text Text Text Text
%\newpage

\tableofcontents\thispagestyle{empty}
\newpage

\pagenumbering{arabic}
\setcounter{page}{1}

% Falls die Kapitel"uberschriften zu lang f"ur die Kopfzeile oder das Inhaltsverzeichnis sind, so erzielt man
% dort Kurzformen der Kapitelbezeichnungen mittels:
% \chapter[Kurzform]{Lange "Uberschrift}
\chapter{Einf"uhrung}

\chapter{Warum Testen}
\cite{eval_automat_webapp_test}[S. 17]
\cite{eval_automat_webapp_test}[S. 35]
\cite{test_large_systems}[S. 10]
\cite{test_auto}[S. 22]
\cite{betrieb}[S. 15]
\cite{eval_regression}[S. 9]

\chapter{Unterschiede zu klassischer Software}
Im Gegensatz zu klassischen Desktop- oder Mobil-Applikationen bestehen Web-Applikationen wegen ihrer Client-Server Architektur immer aus mehreren Modulen, die meist "uber Netzwerk miteinander verbunden sind, beispielsweise: 

\begin{itemize}
\item Datenbank-Server
\item Web-Server
\item Applikations-Server
\item Authentifizierungs-Server
\item Web-Browser
\end{itemize}

Durch diesen Modularen Aufbau ist es besonders schwer einen Fehler zu lokalisieren. Der Fehler k"onnte sich z.B. im Applikations-Code befinden aber auch durch ein Netzwerkproblem entstehen. \cite{testing_apps_on_web}[Foreword]

Au"serdem gibt es eine gr"o"sere Vielfalt an Plattformen, auf welchen die Web-Applikation ausgef"uhrt wird: Auf der Serverseite sind diese Plattformen noch vom/von der BetreiberIn festlegbar, sprich welches Betriebssystem und welche Datenbank eingesetzt wird, auf der Clientseite ist dies aber schon nicht mehr m"oglich. Die BesucherInnen der Webseite verwenden verschiedene Web-Browser auf verschiedenen Betriebssystemen, welche beide in unterschiedlichen Versionen vorliegen k"onnen. Auch k"onnen unterschiedliche Plugins und Fonts in unterschiedlichen Versionen installiert sein. \cite{testing_apps_on_web}[Foreword]

Zudem verlagert sich auch immer mehr Applikations-Logik von der Server- auf die Clientseite\cite{testing_apps_on_web}[S. 13]. Durch das Verwenden von \emph{Events}, welche unter anderem durch BenutzerInnen-Eingaben ausgel"ost werden k"onnen, stellt clientseitiger Code eine gr"o"sere Herausforderung f"ur den/die TesterIn dar als Serverseitiger: Events k"onnen in unterschiedlicher Reihenfolge und Kombination auftreten, manche Aktionen l"osen sogar mehrere Events aus. \cite{testing_apps_on_web}[S. 18]

Eine weitere Herausforderung stellt das Instanzmodell von Web-Applikationen dar: die meisten Web-Applikationen erlauben durch das Verwenden von \emph{Cookies} mehrere Instanzen der Applikation, die jedoch unter der gleichen Session ausgef"uhrt werden. Dies kann zu Synchronisationsproblemen zwischen den einzelnen Instanzen f"uhren, z.B. kann in einer Instanz ein Eintrag gel"oscht werden, der durch eine fehlerhafte Synchronisation f"ur die andere Instanz jedoch noch immer existiert und in weiterer Folge zu Fehlern f"uhren kann. \cite{testing_apps_on_web}[S. 20]

Diese Vielfalt an verschiedenen, m"oglichen Konfigurationen und Herausforderungen erfordert eine neue Herangehensweise an das Thema Software-Test: Die bestehenden Techniken sind \glqq zwar auch notwendig, aber nicht ausreichend, um die Qualit"at der Applikation sicherzustellen\glqq\\ \cite{eval_automat_webapp_test}[S. 18]



\chapter{Testplan}
Planning + Documentation, testplan\cite{testing_apps_on_web}[S. 34]
\cite{test_auto}[S. 3]
\chapter{Unit Test}
\chapter{Integration Test}
\cite{process_oop}[S. 30]
\chapter{Acceptance Test}

\cite{test_auto}[S. 11]

Development acceptance tests:
* Release acceptance tests (smoke tests): mainstream data + mainstream funktionen \cite{testing_apps_on_web}[S. 36]
* Functional acceptance simple test: each dev release to test accessibility of key features on minimum configuration, no full functionality test (file saving example) \cite{testing_apps_on_web}[S. 37-38]

Deployment acceptance tests:
Full installation + configurations
* Task-Oriented Functional Test: Features test (gherkin) against requirements, specs + design docs \cite{testing_apps_on_web}[S. 42]
* Forced Error Test: testing for failures 
* boundary test: test extreme inputs
* system level test: test whole application \cite{testing_apps_on_web}[S. 43]
* real world user level test: echte leute testen um fehler zu finden die man sonst übersieht
* exploratory test: überlegen wo es probleme geben könnte und dort testen
* stress test: limited resource conditions (memory, diskspace, network bandwidth)
* Performance tests: testen wieviel das system verträgt
* Regression tests: für bugs die schonmal aufgetreten sind \cite{testing_apps_on_web}[S. 44]
* Compability + config tests: testen auf unterschiedlichen platformen
* documentation test: testen von zb shortcuts \cite{testing_apps_on_web}[S. 45]
* install uninstall test
* UX tests
* External beta tests \cite{testing_apps_on_web}[S. 46]
* Secuirty tests
* Unit tests

%\section[Erster Abschnitt]{"Uberschrift des ersten Abschnitts}



%\subsection[Erster Unterabschnitt]{"Uberschrift des ersten Unterabschnitts}



%\subsubsection[Erster Unter-Unterabschnitt]{Und noch eine Ebene tiefer} 


%\\[2\baselineskip]
%Hier wird auf Abbildung~\ref{Abb1} verwiesen. 
%\begin{figure}[htbp]
%\centering
%\includegraphics[width=75mm]{Buchruecken}
%\caption[Beschriftung eines Buchr"uckens.]{Beispiel f"ur die Beschriftung eines
%Buchr"uckens.}\label{Abb1}
%\end{figure}
%%Tabelle~\ref{Tab1} ist ein Beispiel daf"ur, wie eine Tabelle aussehen k"onnte.
%\begin{table}[htbp]
%\centering
%\begin{tabular}{ | c | c | c | }\hline
%{\bf Datum} & {\bf Thema} & {\bf Raum}\\ \hline
%\hline
%20. 08. 2008 & Graphentheorie & HS 3.13\\ \hline
%01. 10. 2008 & Biomathematik & HS 1.05\\ \hline
%\end{tabular}
%\caption[Semesterplan "`Angewandte Mathematik"'.]{Beispiel f"ur einen
%Semesterplan "`Angewandte Mathematik"'.}\label{Tab1}
%\end{table}

%\noindent
%Nun ein Beispiel f"ur eine abgesetzte Formel:
%\begin{equation}
%x =  - \frac{p}{2} \pm \sqrt{\left(\frac{p}{2}\right)^2 - q}.
%\end{equation}
%Und eine mehrzeilige Formel:
%\begin{eqnarray}
%f(t)&=& t^2 \label{For1},\\
%g(t) &=& t-1.
%\end{eqnarray}
%Hier wird auf die Formel (\ref{For1}) verwiesen. \\

%\noindent
%So kann zum Beispiel ein \glqq Source-Code\grqq\  angegeben werden: 
%\begin{verbatim}
%for (i=1; i < 10; i++) {...} 
%\end{verbatim}

%\noindent
%Hier ist ein Hyperlink auf die  \href{http://www.technikum-wien.at}{Homepage}
%der FH Technikum Wien. Email-Adressen k"onnen so verlinkt werden:
%\href{mailto:homer.simpson@springfield.com}{\texttt{
%homer.simpson@springfield.com}}\\

%\noindent
%In der Bibliothek der Fachhochschule Technikum Wien gibt es verschiedene
%einf"uhrende B"ucher zum Thema \glqq \LaTeX \grqq, zum Beispiel \cite{kop05},
%\cite{wil06} oder \cite{mgb+05d} (deutsche Version) bzw. \cite{mgb+04e}
%(englische Version). Empfehlenswerte Skripten f"ur \LaTeX-Einsteiger sind z.B.
%\cite{mj00} und \cite{mj95}. Sie sind frei im Internet verf"ugbar.



% Literaturverzeichnis
% Das Literaturverzeichnis kann auch nach einem allf"alligen Anhang positiioniert werden (siehe "`Leitfaden f"ur Bachelor- und Diplomarbeiten"', Version 2.0, Abschnitt 2.9).

% M"oglichkeit 1: Erzeugung des Literaturverzeichnisses mit BibTeX:
% Die Quellen sind in der Datei *.bib (hier Literatur.bib) einzugeben. Danach muss diese Vorlage einmal geTeXt werden, dann BibTeX angewendet werden und 
% anschliessend nochmals zweimal geTeXt werden.
% Im Text erfolgt die Zitierung mit dem Anker-Schl"usselwort, z.B. \cite{kop05}.
\bibliographystyle{IEEEtran}
\bibliography{Literatur}

% M"oglichkeit 2: Erzeugung eines Literaturverzeichnisses ohne BibTeX:
%\begin{thebibliography}{99}
%\bibitem[kop05]{kop05}
%H.~Kopka, {\em LaTeX, Band 1: Einf"uhrung}, Pearson Studium, M"unchen, 3.~Auflage, 2005.
%\bibitem[knu98]{knu98}
%F.~Mittelbach, M.~Goossens, J.~Braams, D.~Carlisle, and Ch. Rowley, {\em The LaTeX Companion}, 
%Addison-Wesley, 2nd edition, 2004.
%\end{thebibliography}

% Abbildungsverzeichnis
\listoffigures
\addcontentsline{toc}{chapter}{Abbildungsverzeichnis} % f"ugt den Eintrag "Abbildungsverzeichnis" im Inhaltsverzeichnis hinzu
\newpage

% Tabellenverzeichnis
%\listoftables 
%\addcontentsline{toc}{chapter}{Tabellenverzeichnis} % f"ugt den Eintrag
%"Tabellenverzeichnis" im Inhaltsverzeichnis hinzu
%\newpage

% Abk"urzungsverzeichnis
% Bei Verwendung der Dokumentklasse "scrartcl" ist der Befehlt \addchap{Abk"urzungsverzeichnis} durch 
% \addsec{Abk"urzungsverzeichnis} zu ersetzen
\addchap{Abk"urzungsverzeichnis}
\hspace{-17mm}\begin{tabular}{>{\raggedleft}p{0.2\linewidth} p{0.75\linewidth} p{0.1\linewidth}}

% www & World Wide Web\\

\end{tabular}

% Anh"ange
%\begin{appendix}
%\chapter[Erster Anhang]{"Uberschrift des ersten Anhangs}

%Text Text Text Text Text Text Text Text Text Text Text Text Text Text Text Text
%\end{appendix}

\end{document}
